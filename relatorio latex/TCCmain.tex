\documentclass[12pt,a4paper,twoside,openright]{report}

\usepackage[table,xcdraw]{xcolor}
\usepackage{array}
\usepackage{colortbl}

\usepackage[pdftex]{graphicx} % Biblioteca para uso de figuras
\usepackage{color}

\usepackage[brazil]{babel} % Biblioteca para uso da l�ngua portuguesa
\usepackage[T1]{fontenc} % Biblioteca para uso da acentua��o de entrada
\usepackage[latin1]{inputenc} % Biblioteca para uso da acentua��o de sa�da

\usepackage{indentfirst}

\usepackage{caption}

\usepackage{subcaption}

\usepackage{cleveref}

\usepackage[export]{adjustbox}





\usepackage{amsthm,amsfonts,amsmath,amssymb}  % Biblioteca para uso de comandos matem�ticos
\usepackage{pslatex}
\usepackage{pstricks,pst-node,color,pst-gantt,pst-coil}
\usepackage{scalefnt}
\usepackage{float} %Permite colocar "\begin{figure}[H]" e colocar imagem exatamente onde desejar
\usepackage[hyphens]{url} % Para Aceitar URL nas referencias
\usepackage{pdfpages}

\usepackage{eurosym} %Pacote para possibilitar o uso do s�mbolo de euro "\euro"

\usepackage{listings} % para importa��o de c�digos fonte 

\usepackage{setspace}

\usepackage{comment}

%\usepackage[margin=2.7cm]{geometry}
\renewcommand{\baselinestretch}{1.5}

\setlength{\parskip}{0em}

% Pacote para configurar cabe�alho e rodape
\usepackage{fancyhdr}
\pagestyle{empty}
\fancyhf{} % clear all header and footer fields

\fancypagestyle{plain}{\pagestyle{fancy}}
\renewcommand{\headrulewidth}{0pt}
\renewcommand{\footrulewidth}{0pt}

%Pacote para organizar apendices 
\usepackage[titletoc]{appendix}

\usepackage{rotating}

%%%%%%%%%%%%%%%%%%%%%%%%%%%%%%%%%%%%%% CONFIGURA��ES DE ANEXOS %%%%%%%%%%%%%%%%%%%%%%%%%%%%%%%%%%%%%%

\newcommand{\annexname}{Anexo}
\makeatletter
\newcommand\annex{\par
  \setcounter{chapter}{0}%
  \setcounter{section}{0}%
  \gdef\@chapapp{\annexname}%
  \gdef\thechapter{\@Roman\c@chapter}}
\makeatother

\newenvironment{poliabstract}[1]
  {\renewcommand{\abstractname}{#1}\begin{abstract}}
  {\end{abstract}}

%%%%%%%%%%%%%

\usepackage{mathptmx}
\usepackage{helvet}
\renewcommand{\familydefault}{\sfdefault}
	
 %\usepackage[T1]{fontenc}
 %\usepackage[scaled]{uarial}
 %\renewcommand*\familydefault{\sfdefault}	
	
	%\pdfmapfile{=<name>.map}
	
\renewcommand{\baselinestretch}{1.5} 	

\setlength\parindent{1cm}

\let\locdimen\newdimen

\usepackage{etex}

\usepackage{bytefield}

\usepackage{titlesec}
\titlespacing*{\chapter}{0pt}{-0.5cm}{20pt}

\titleformat{\chapter}{\fontsize{16pt}{5mm}\selectfont\bfseries}{\hspace{1cm}\thechapter}{1em}{}
\titleformat{\section}{\fontsize{13.5pt}{5mm}\selectfont\bfseries}{\hspace{1cm}\thesection}{1em}{}
\titleformat{\subsection}{\fontsize{13.5pt}{5mm}\selectfont\bfseries}{\hspace{1cm}\thesubsection}{1em}{}

\usepackage[top=3cm, bottom=2cm, left=3cm, right=2cm]{geometry}


%\usepackage{titlesec}
%\titleformat{\section}[display]
%{\hspace{1cm}\fontsize{12pt}{5mm}\selectfont\bfseries}{\thesection}{1em}{}

%\usepackage{titlesec}
%\titleformat{\subsection}
%{\hspace{1cm}\fontsize{12pt}{5mm}\selectfont\bfseries}{\thesubsection}{1em}{}

%%%%%%%%%%%%%%%%%%%%%%%%%%%%%%%%%%%%%% CONFIGURA��ES DE P�GINA %%%%%%%%%%%%%%%%%%%%%%%%%%%%%%%%%%%%%%
%\topmargin -2.1cm
%\oddsidemargin 0.5cm 
%\evensidemargin 0.5cm 
%\textwidth 15cm
%\textheight 25.1cm

%\usepackage{hyperref}


%%%%%%%%%%%%%%%%%%%%%%%%%%%%%%%%%%%%%%%% IN�CIO DO DOCUMENTO %%%%%%%%%%%%%%%%%%%%%%%%%%%%%%%%%%%%%%%%
\begin{document}

%%%%%%%%%%%%%%%%%%%%%%%%%%%%%%%%%%%%%%  INCLUDES %%%%%%%%%%%%%%%%%%%%%%%%%%%%%%%%%%%%%%

%%%%%%%%%%%%%%%%%%%%%%%%%%%%%%%%%%%%%% ELEMENTOS PR�-TEXTUAIS %%%%%%%%%%%%%%%%%%%%%%%%%%%%%%%%%%%%%%
%%%%%%%%%%%%%%%%%%%%%%%%%%%%%%%%%%%%%%% CONFIGURA??ES DE CAPA %%%%%%%%%%%%%%%%%%%%%%%%%%%%%%%%%%%%%%%

\begin{titlepage}
	
	% CAPA PRINCIPAL
	\begin{center}
		\huge{UNIVERSIDADE DE S�O PAULO}\\
		\vspace{0.02\textheight}
		\LARGE{ESCOLA DE ENGENHARIA DE S�O CARLOS}\\
		\vspace{0.03\textheight}
		\large{DEPARTAMENTO DE ENGENHARIA EL�TRICA E DE COMPUTA��O}\\
		\vspace{0.08\textheight}
		\huge{\textbf{DESENVOLVIMENTO DE SISTEMA SEGURO DE COMUNICA��O}}\\
		\vspace{0.05\textheight}
		\large{Guilherme Galdino Siqueira}\\
		\large{Relat�rio de est�gio}
		
		\vspace{0.10\textheight}
	\end{center}
	
	
	
	\raggedleft
	\begin{table}[H]
		\large
		\begin{tabular}{ll}			
			\textbf{Gestores:}  	& Diaulas Gonzaga   \\
									& Paulo Cesar Pires \\
									&\\
			\textbf{Mentor:}  		& Evandro Joselito Carrenho   \\
									&\\									
			\textbf{Orientador:}    & Prof. Evandro Luis Linhari Rodrigues    \\			
		\end{tabular}
	\end{table}
			
	\begin{center}
		\vfill{\large{S�o Carlos\\2016}}
	\end{center}
		
	
\end{titlepage}


%%%%%%%%%%%%%%%%%%%%%%%%%%%%%%%%%%%%%%%%%%%%%%% INSER??O P?GINA EM BRANCO %%%%%%%%%%%%%%%%%%%%%%%%%%%%%%%%%%%%%%%%%%%%%%


%%%%%%%%%%%%%%%%%%%%%%%%%%%%%%%%%%%%%%%%%%%%%%% RESUMO - PORTUGUES %%%%%%%%%%%%%%%%%%%%%%%%%%%%%%%%%%%%%%%%%%%%%%
%%\
%%\vspace{0.11\textheight} 

\begin{center}
	\fontsize{16pt}{21pt}\selectfont\bfseries{\textbf{Resumo}}	
\end{center}

O est�gio foi realizado na �rea de desenvolvimento da Daitan Group, empresa com atua��o nos mais diversos ramos de TI (Tecnologia da Informa��o). Foram realizadas tarefas de programa��o majoritariamente em linguagem Java para implementa��o de m�dulos de integra��o entre servi�os sobre um sistema de comunica��o com seguran�a de informa��o.




\vspace{0.05\textheight}
	
Palavras-Chave: .

%%%%%%%%%%%%%%%%%%%%%%%%%%%%%%%%%%%%%%%%%%%%%%% INSER??O P?GINA EM BRANCO %%%%%%%%%%%%%%%%%%%%%%%%%%%%%%%%%%%%%%%%%%%%%%



%%%%%%%%%%%%%%%%%%%%%%%%%%%%%%%%%%%%%%%%%%%%%%% RESUMO - INGL�S %%%%%%%%%%%%%%%%%%%%%%%%%%%%%%%%%%%%%%%%%%%%%%



%%%%%%%%%%%%%%%%%%%%%%%%%%%%%%%%%%%%%%%%%%%%%%% INSER??O P?GINA EM BRANCO %%%%%%%%%%%%%%%%%%%%%%%%%%%%%%%%%%%%%%%%%%%%%%


%\thispagestyle{empty}
%\newpage
%%%%%%%%%%%%%%%%%%%%%%%%%%%%%%%%%%%%%%%%%%%%%%% RESUMO %%%%%%%%%%%%%%%%%%%%%%%%%%%%%%%%%%%%%%%%%%%%%

%%%%%%%%%%%%%%%%%%%%%%%%%%%%%%%%%%%%% CONFIGURA??ES DOS ?NDICES %%%%%%%%%%%%%%%%%%%%%%%%%%%%%%%%%%%%%
%\clearpage
%\thispagestyle{empty}

\listoffigures % ?ndice de Figuras
\listoftables % ?ndice de Tabelas

%%%%%%%%%%%%%%%%%%%%%%%%%%%%%%%%%%%%%%%%%%%%%%% INSER??O P?GINA EM BRANCO %%%%%%%%%%%%%%%%%%%%%%%%%%%%%%%%%%%%%%%%%%%%%%
\cleardoublepage


%%%%%%%%%%%%%%%%%%%%%%%%%%%%%%%%%%%%% LISTA DE ABREVIATURAS %%%%%%%%%%%%%%%%%%%%%%%%%%%%%%%%%%%%%


 

\begin{titlepage}
{
	\vspace{0.11\textheight}
	\raggedleft%
	\textbf{\fontsize{16pt}{5mm}\selectfont\bfseries{Siglas}}
	\vspace{0.05\textheight}
}
{
	\begin{tabbing}
		\hspace*{0.5cm}\=\hspace{2.5cm}\= \kill
		
		% Exemplo de lista de lista de abreviaturas
		\> API \> \textit{Application Programming Interface} - Interface de Programa��o de Aplicativos \\
		\> OCR \> \textit{Optical Character Recognition} - Reconhecimento �tico de Caracter \\
		\> TTS \> \textit{Text-To-Speech} - Texto-Para-Fala \\
		\> IoT \> \textit{Internet of Things} - Internet das Coisas \\
		\> IDE \> \textit{Integrated Development Environment} - Ambiente de Desenvolvimento Integrado \\
		\> ETA \> \textit{Electronic Travel	Aid} - Subs�dio Eletr�nico de Percurso \\
		\> SaaS \> \textit{Software as a Service} - Software como Servi�o \\
		
		
	\end{tabbing}
}
\end{titlepage}


			

\cleardoublepage
%%%%%%%%%%%%%%%%%%%%%%%%%%%%%%%%%%%%% CONFIGURA��ES DOS �NDICES %%%%%%%%%%%%%%%%%%%%%%%%%%%%%%%%%%%%%
%\usepackage{fancyhdr}


\pagestyle{fancy}
\fancyhf{} % clear all header and footer fields
\fancyhead[RO, LE] {\thepage}

\fancypagestyle{plain}{\pagestyle{fancy}}




\setcounter{page}{19}
\tableofcontents % �ndice Geral




%%%%%%%%%%%%%%%%%%%%%%%%%%%%%%%%%%%%%%%% ADI��O DOS CAP�TULOS %%%%%%%%%%%%%%%%%%%%%%%%%%%%%%%%%%%%%%%	
\chapter{Introdu��o}

\label{Introducao}

% % % % % % % % % % % % % % % % % % % % % % % % % % % % % % % % % % % % % % % % % % % % % % % % % % %
\section{Objetivos}

Este trabalho tem como objetivo apresentar as atividades realizadas durante o est�gio na Daitan Groups, levando em considera��o o aprendizado obtido com a experi�ncia de trabalho no mercado e a utiliza��o do conhecimento obtido na universidade.

% % % % % % % % % % % % % % % % % % % % % % % % % % % % % % % % % % % % % % % % % % % % % % % % % % %

\section{Organiza��o do trabalho}	

\begin{itemize}
	\item Cap�tulo \ref{atividades}: Descreve o conjunto de atividades realizadas, sejam elas relacionadas ao conhecimento da empresa e ao conv�vio no  mundo corporativo, sejam relacionadas ao trabalho desempenhado na fun��o de estagi�rio e o cronograma.
	\item Cap�tulo \ref{Resultados}: Apresenta os resultados obtidos durante os primeiros meses de trabalho, em termos de aquisi��o de novos conhecimentos e experi�ncias.
	\item Cap�tulo \ref{Conclusao}: Resume os principais pontos de todo o progresso no est�gio, explorando os ganhos e as dificuldades encontradas para avaliar sua contribui��o na forma��o em Engenharia de Computa��o.
\end{itemize}


 
\section{A empresa}

A Daitan foi fundada em 2004 com ra�zes profundas na industria de telecomunica��es e j� liderou o desenvolvimento e a implanta��o de solu��es em mais de 50 pa�ses. A empresa atualmente se especializa mo mercado de Pesquisa e Desenvolvimento (P\&D) e Servi�os de Engenharia nos setores de Telecomunica��es e TI.

A empresa possui duas sedes, uma administrativa em San Ramon na Calif�rnia, e outra de desenvolvimento na cidade de Campinas.




\subsection{Miss�o}

Ser refer�ncia mundial em empreendedorismo, inova��o e excel�ncia em execu��o.

Transformar conhecimento em oportunidades de neg�cio, de forma a agregar valor para clientes, acionistas, colaboradores e comunidades em que atuamos.


\subsection{Campo de atua��o}

Sua base de clientes � formada por empresas desenvolvedoras de tecnologias de ponta fornecedoras de servi�os e produtos nas �reas de Telecomunica��es, Tecnologia da Informa��o, Computa��o em Nuvem, Redes Sociais e Jogos. Essas empresas, espalhadas pelo mundo, compreendem desde os Estados Unidos, at� pa�ses da Europa e �sia.




\subsection{O projeto}

A Daitan j� possuiu como clientes as mais diversas empresas, inclusive grandes nomes como a Microsoft. Devido entretanto a restri��es de sigilo empresarial, n�o ser�o apresentadas informa��es sobre a empresa contratante do projeto trabalhado, apenas uma breve descri��o. 

O projeto consistiu, de forma superficial, em um sistema de troca de mensagens entre funcion�rios de uma empresa, com a particularidade de realiz�-la de forma segura, protegendo os dados trocados, e integrando diversos m�dulos de servi�os online para centralizar no mesmo sistema notifica��es de uso, al�m de permitir v�deo-chamadas como alternativa � troca de mensagens em texto.

\subsection{Estrutura organizacional}



\subsection{Metodologia de desenvolvimento}

A metodologia utilizada para o desenvolvimento do sistema foi o Scrum. A filosofia por tr�s dessa metodologia\cite{scrum} � que ela enfatiza tomadas de decis�es baseadas no mundo real, e n�o em especula��o. O tempo � dividido em pequenas partes, cerca de uma ou duas semanas, conhecidas como Sprints, que englobam as tarefas a serem realizadas, conhecidas como tickets. O produto � mantido propriamente integrado e testado em todos os tempos. Ao final de cada Sprint, as partes interessadas e os membros da equipe se encontram para avaliar o incremento no produto, j� integrado e testado, e ent�o decidir os pr�ximos passos.

Devido a dimens�o do projeto, o time completo apenas da Daitan nesse projeto possu�a mais de 20 pessoas. Para haver uma boa ger�ncia, o time foi quebrado em grupos menores, com cerca de 6 pessoas em m�dia, contendo desenvolvedores e um Tech Lead, pessoa respons�vel por organizar as reuni�es com os membros, estabelecer os tickets a cada um, discutir o andamento do projeto, e se comunicar com o gerente e com o cliente.
\chapter{Atividades Realizadas}
\label{atividades}

Neste cap�tulo, ser�o apresentadas todas as atividades realizadas durante os primeiros 3 meses de est�gio.

\section{Apresenta��o da empresa}

J� no primeiro dia de trabalho, o departamento de recursos humanos estabeleceu um pequeno cronograma de atividades para a familiariza��o com a empresa. Foram marcadas duas reuni�es de apresenta��o e requisitado que o gestor de projetos marcasse um almo�o de integra��o com novos funcion�rios.

\subsection{Reuni�es de Apresenta��o}

Na primeira reuni�o, ministrada pelo vice-presidente de opera��es, a empresa foi apresentada aos novos funcion�rios atrav�s da sua hierarquia, clientes e projetos.

Na segunda reuni�o, foram apresentados os programas de desenvolvimento pessoal e profissional que empresa oferece a seus funcion�rios, os benef�cios, e os deveres, com foco nas regras de sigilo empresarial e uso da rede dentro da empresa.

\subsection{Almo�o de boas vindas}

Organizado pelos gestores e arcado pela empresa, o almo�o de integra��o foi realizado com a finalidade de esclarecer os novos funcion�rios eventuais d�vidas do trabalho e integrar o time.

\section{Configura��o do ambiente}

No primeiro dia de trabalho, o departamento de TI disponibilizou um computador da bancada e uma conta de acesso do Windows 10. Como o desenvolvimento do projeto vinha sendo feito por outros sistemas operacionais, foi instalado a vers�o mais recente do Ubuntu para o in�cio das configura��es do ambiente de desenvolvimento. 

Baseado em um tutorial de instala��es, produzido por um dos membros do time, foi instalada uma m�quina virtual que futuramente seria utilizada para permitir a realiza��o de testes do sistema localmente. Ent�o foi instalada a IDE de desenvolvimento IntelliJ Idea e o servidor apache Tomcat.

Por fim, o projeto armazenado em uma conta do Git Hub da empresa foi clonado para uma pasta de trabalho e, com isso, foi poss�vel iniciar um avan�o em cima propriamente do projeto.


\section{Estudo do sistema}

Antes de iniciar qualquer altera��o sobre o c�digo, foi extremamente importante compreender como o sistema funcionava, ou no m�nimo as mais necess�rias devido a dimens�o do projeto, muito maior que qualquer trabalho realizado na universidade. Para isso foi disponibilizados no sistema de gerenciamento da empresa diagramas de fluxo das partes mais importantes, al�m da disponibilidade tanto do Tech Lead, quanto de todos os membros para explicar qualquer ponto do projeto.

\section{Cria��o de c�digo}

Ap�s um pequeno estudo do projeto, foram atribu�dos os primeiros tickets do sprint relacionados ao backend do sistema. Inicialmente foram tarefas muito simples de corre��es de bugs, depois algumas tarefas n�o muito dif�ceis de inser��o de funcionalidade nos m�dulos de integra��o de servi�os, ent�o algumas um pouco mais complexas e repetitivas de constru��o de novos m�dulos, cuja estrutura era parecida com de outros j� implementados.

Com certa pr�tica obtida, foram requisitados tickets relacionados ao frontend do sistema. Novamente, tarefas simples de corre��es como exibi��o correta de caracteres especiais, at� algumas um pouco mais elaboradas e repetitivas de mudan�as no conte�do de todas mensagens exibidas por todos os m�dulos do sistema.

Foram realizadas algumas tarefas que envolviam a investiga��o de causa de erros, um pouco mais complexas devido novamente a dimens�o do c�digo, que permitiram por consequ�ncia maior compreens�o sobre ele.



\section{Reuni�es}

Ao t�rmino de cada sprint, cuja dura��o era de 2 ou 3 semanas, uma reuni�o entre os membros do time e o Tech Lead eram realizadas a fim de avaliar o desenvolvimento do projeto, checar o cumprimento dos tickets, avaliar poss�veis erros cometidos e buscar solu��es, estabelecer a pontua��o de complexidade dos tickets do sprint seguinte e redistribu�-los pelos membros.

\section{Dificuldades}
Durante todo o tempo de est�gio realizado, muitas dificuldades foram encontradas. O fato de se sair da universidade, em que se realiza provas escritas e se trabalha no m�ximo com pequenos projetos e entrar em uma empresa com projetos de enormes dimens�es, ser requisitado a compreender o funcionamento de um sistema complexo e nunca visto com implementa��o em andamento foi o principal dificultador.
 
Outro fator dificultador foram as configura��es de ambiente. Devido � grande quantidade de passos a serem seguidos, algumas vezes n�o muito claros, ocorreu uma atraso para o t�rmino dessa etapa, que inclusive propagou problemas inicialmente n�o identificados a v�rias etapas seguintes.

Ao in�cio de cada nova atividade, houve sempre certa dificuldade de compreens�o do problema a ser resolvido e como seria feito. A resolu��o de bugs simples do sistema dependia de uma ampla compreens�o da integra��o dos m�dulos do sistema, algo que custava tempo a se conseguir, por�m, uma vez compreendido, custava cada vez menos tempo para tarefas relacionadas.


\section{Rela��o com o curso}

Se h� uma disciplina que se relaciona diretamente com as atividades realizadas durante o est�gio, essa disciplina � SSC0620 - Engenharia de Software. A raz�o � simples, pela primeira vez trabalhando sobre um projeto real, de grandes propor��es, era necess�rio que se seguisse alguma metodologia de desenvolvimento ou n�o seria realiz�vel. A metodologia �gil utilizada, Scrum, serviu de aplica��o direta de conhecimentos adquiridos em parte da disciplina. 

\section{Calend�rio}



\chapter{Conclus�o}
\label{Conclusao}

Esse projeto teve como objetivo a aplica��o dos conhecimentos adquiridos durante a gradua��o a um problema real que nem sempre recebe a devida aten��o de profissionais da �rea. Esse problema � o da acessibilidade de pessoas com defici�ncia visual. Os resultados mostraram que a API Google Cloud Vision � uma ferramenta poderosa no campo de vis�o computacional e seus servi�os podem ser utilizados pelas mais diversas aplica��es, incluindo a possibilidade de auxiliar pessoas sem o sentido visual a acessar informa��es visuais com autonomia. Ser� apresentado neste cap�tulo um balan�o dos resultados obtidos atrav�s dos testes, a fim de avaliar sua utilidade a aplicabilidade.

\section{Limita��es}

Apesar de a API da Google ter se mostrado eficiente para o prop�sito do projeto, a ferramenta causou algumas limita��es para os resultados. Um deles � o fato intr�nseco de que a API est� implementada para operar em nuvem e requer conex�o com a internet. Isso poderia de certa forma impedir que pessoas sem acesso a internet pudessem us�-lo. Entretanto, existe uma grande tend�ncia de os dispositivos e as pessoas se tornarem cada vez mais conectadas, devido o avan�o da computa��o e das telecomunica��es, e com os pre�os de smartphones mais acess�veis, e isso poderia preparar um ambiente mais prop�cio para o uso do aplicativo. Outro ponto negativo � a limita��o da gratuidade de utiliza��o da API. Com o intuito do projeto sendo tamb�m o de inclus�o digital, atribuir pre�o para o acesso ao aplicativo iria, de certa forma, contra esse prop�sito.

J� com rela��o ao circuito detector de obst�culos, um dos problemas foi o consumo de energia pelo m�dulo Bluetooth, que representou quase a metade do consumo total. Num cen�rio em que a pessoa o utilize sempre ao caminhar, a bateria  poderia descarregar com certa rapidez. Al�m disso, a utiliza��o de um �nico sensor foi capaz apenas de mostrar que � poss�vel se guiar por ele, entretanto, para um caminhar mais seguro, seriam necess�rios mais sensores apontando para v�rias dire��es. Tamb�m, o tamanho do dispositivo se mostrou relativamente grande para ser utilizado na parte frontal do corpo, o que poderia dificultar a usabilidade.

N�o foi poss�vel, devido � indisponibilidade de mais recursos, a produ��o de testes mais elaborados com um n�mero maior de vers�es do Android, e tamb�m, testes de usabilidade com usu�rios para o qual o projeto � dedicado, o que permitiria sua valida��o. Por essa raz�o, n�o � poss�vel concluir se o sistema de fato atende as necessidades desses potenciais usu�rios.

\section{Acertos}


Por outro lado, houve v�rios pontos positivos que se podem extrair deste projeto. A leitura de textos, que � a funcionalidade mais importante do sistema mostrou-se acurada. Com ela seria poss�vel a leitura de r�tulos de produtos, folhetos, livros e at� bulas de rem�dio, mesmo possuindo letras pequenas, cupons fiscais, dinheiro, entre diversos outros conte�dos escritos. Com o aplicativo � poss�vel tamb�m oferecer uma descri��o superficial autom�tica do ambiente ao redor e de express�es faciais, e permite at� que o usu�rio insira os nomes das pessoas em uma foto. 

Com rela��o ao detector de obst�culos, apesar da limita��o do n�mero de sensores, os resultados se mostraram precisos, e o dispositivo tem o potencial de se tornar ainda mais �til e confi�vel caso o n�mero de sensores seja aumentado e seu tamanho reduzido.

\section{Opini�o de potenciais usu�rios}

Durante o per�odo de desenvolvimento do projeto foi importante procurar ouvir um pouco as pessoas para as quais o projeto se destinaria. Assim, os membros do grupo do Facebook "Cegos e a Tecnologia", em sua maioria com algum grau de defici�ncia visual, deram algum suporte apresentando algumas de suas dificuldades, ou sugerindo funcionalidades. 

Um dos problemas citados foi a dificuldade de posicionamento da c�mera para utiliza��o de aplicativos para descri��o de valor de dinheiro, e o tempo gasto na procura pela posi��o ideal. Nesse quesito, pode se dizer que houve sucesso, uma vez que a funcionalidade de descri��o de textos mostrou-se precisa para v�rios �ngulos. Outro ponto levantado foi a sugest�o de implementa��o do comando de voz. A ideia da descri��o de express�es faciais foi considerada de modo geral �til, assim como a possibilidade de guardar suas informa��es junto �s fotos tiradas. Uma preocupa��o apresentada foi o tamanho do detector de obst�culos que dificultaria seu uso e a limita��o do raio de abrang�ncia do sensor.

Entre sugest�es e cr�ticas, os membros do grupo se mostraram contentes com um projeto que pudesse vir a auxili�-los, o que � extremamente gratificante e incentiva a continuidade do trabalho, algo que segundo eles pr�prios, muitas vezes n�o ocorre ap�s estudantes extra�rem informa��es deles e seus projetos serem apresentados.

\section{Disciplinas base}

Foi poss�vel por meio desse trabalho aplicar conceitos de algumas das disciplinas do curso de Engenharia de Computa��o. Essas disciplinas s�o apresentadas na lista a seguir e as que est�o marcadas com (*) foram cursadas na Northern Arizona University - EUA, durante interc�mbio pelo programa Ci�ncia Sem Fronteiras.

\begin{itemize}
	\item Laborat�rio de F�sica: M�todos de medidas f�sicas e amostragem.
	\item Circuitos El�tricos: Modelagem de circuitos el�tricos.
	\item Engenharia de software: Extra��o de requisitos, planejamento de testes e codifica��o do sistema.
	\item Programa��o Orientada a Objetos: Conceitos de orienta��o a objetos e linguagem Java.
	\item Estat�stica: Amostragem e c�lculo de incerteza
	\item Laborat�rio de Circu�tos Eletr�nicos: Medidas de vari�veis el�tricas e constru��o de circuitos eletr�nicos.
	\item Engineering Design - The Process*: Planejamento e implementa��o de projetos de automa��o utilizando Ardu�no e sensores.
	\item Pattern Recognition*: Projetos e aplica��es de reconhecimento de padr�o.
	\item Multim�dia e Hiperm�dia: Codifica��o em XML e desenvolvimento de layout.	
	\item Microprocessadores e Aplica��es II: Vis�o computacional, Integra��o entre sistemas embarcados e desenvolvimento Android.
\end{itemize}





\section{Trabalhos futuros}



Com a inten��o de continuidade do projeto, alguns pontos que necessitam seja de melhoria, seja de corre��o, devem ser tratados. 

\begin{itemize}
	\item N�mero de sensores utilizados pelo circuito: Mais sensores permitiriam maior acur�cia e precis�o das dist�ncias retornadas pelo sistema.
	\item Tamanho do dispositivo: A fabrica��o do circuito diretamente em uma placa eletr�nica dedicada, e de tamanho reduzido permitiria maior usabilidade ao usu�rio.
	\item Taxa de uso da API: Do lado do aplicativo, � indispens�vel buscar maneiras de n�o repassar ao usu�rio o valor que � cobrado pela utiliza��o da API, quando o limite � ultrapassado. Uma possibilidade seria aplicar uma solu��o diferente ao problema, deixando de lado a API da Google. A utiliza��o de OpenCV poderia ser uma solu��o mais adequada ao problema, por�m demandaria esfor�os para implementa��o de algoritmos complexos de vis�o computacional. Outra possibilidade seria a inser��o de propaganda como um modo de compensar essa cobran�a.
	\item Integra��o com redes sociais: Com as redes sociais, em especial o Facebook e o SnapChat, desempenhando importante papel na comunica��o entre as pessoas, integrar funcionalidade de compartilhamento das descri��es poderia contribuir ainda mais com a acessibilidade de outras pessoas com defici�ncia visual.
	\item Expans�o para outras plataformas: � importante considerar que existem usu�rios com smartphones rodando sistemas operacionais diferentes do Android. Uma possibilidade seria migrar o aplicativo para essas plataformas.
	\item Elabora��o de testes de usabilidade com usu�rios para avalia��o e melhorias das fun��es existentes, e inser��o de novas, e testes de compatibilidade com as variadas vers�es do Android.
	\item Inser��o de entrada por comando de voz ao aplicativo para tornar a navega��o mais f�cil.
\end{itemize}

 

 







%%%%%%%%%%%%%%%%%%%%%%%%%%%%%%%%%%%%%% ADI��O DAS BIBLIOGRAFIAS %%%%%%%%%%%%%%%%%%%%%%%%%%%%%%%%%%%%%

\cleardoublepage
\addcontentsline{toc}{chapter}{Refer�ncias}
\renewcommand{\bibname}{Refer�ncias}
	
\bibliographystyle{IEEEtran} % Define o estilo da bibliografia
\bibliography{./Content/References} % Faz referencia ao arquivo ref.bib

\end{document}    
	