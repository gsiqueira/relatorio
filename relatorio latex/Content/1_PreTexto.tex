%%%%%%%%%%%%%%%%%%%%%%%%%%%%%%%%%%%%%%% CONFIGURA??ES DE CAPA %%%%%%%%%%%%%%%%%%%%%%%%%%%%%%%%%%%%%%%

\begin{titlepage}
	
	% CAPA PRINCIPAL
	\begin{center}
		\huge{UNIVERSIDADE DE S�O PAULO}\\
		\vspace{0.02\textheight}
		\LARGE{ESCOLA DE ENGENHARIA DE S�O CARLOS}\\
		\vspace{0.03\textheight}
		\large{DEPARTAMENTO DE ENGENHARIA EL�TRICA E DE COMPUTA��O}\\
		\vspace{0.08\textheight}
		\huge{\textbf{DESENVOLVIMENTO DE SISTEMA SEGURO DE COMUNICA��O}}\\
		\vspace{0.05\textheight}
		\large{Guilherme Galdino Siqueira}\\
		\large{Relat�rio de est�gio}
		
		\vspace{0.10\textheight}
	\end{center}
	
	
	
	\raggedleft
	\begin{table}[H]
		\large
		\begin{tabular}{ll}			
			\textbf{Gestores:}  	& Diaulas Gonzaga   \\
									& Paulo Cesar Pires \\
									&\\
			\textbf{Mentor:}  		& Evandro Joselito Carrenho   \\
									&\\									
			\textbf{Orientador:}    & Prof. Evandro Luis Linhari Rodrigues    \\			
		\end{tabular}
	\end{table}
			
	\begin{center}
		\vfill{\large{S�o Carlos\\2016}}
	\end{center}
		
	
\end{titlepage}


%%%%%%%%%%%%%%%%%%%%%%%%%%%%%%%%%%%%%%%%%%%%%%% INSER??O P?GINA EM BRANCO %%%%%%%%%%%%%%%%%%%%%%%%%%%%%%%%%%%%%%%%%%%%%%


%%%%%%%%%%%%%%%%%%%%%%%%%%%%%%%%%%%%%%%%%%%%%%% RESUMO - PORTUGUES %%%%%%%%%%%%%%%%%%%%%%%%%%%%%%%%%%%%%%%%%%%%%%
%%\
%%\vspace{0.11\textheight} 

\begin{center}
	\fontsize{16pt}{21pt}\selectfont\bfseries{\textbf{Resumo}}	
\end{center}

O est�gio foi realizado na �rea de desenvolvimento da Daitan Group, empresa com atua��o nos mais diversos ramos de TI (Tecnologia da Informa��o). Foram realizadas tarefas de programa��o majoritariamente em linguagem Java para implementa��o de m�dulos de integra��o entre servi�os sobre um sistema de comunica��o com seguran�a de informa��o.




\vspace{0.05\textheight}
	
Palavras-Chave: .

%%%%%%%%%%%%%%%%%%%%%%%%%%%%%%%%%%%%%%%%%%%%%%% INSER??O P?GINA EM BRANCO %%%%%%%%%%%%%%%%%%%%%%%%%%%%%%%%%%%%%%%%%%%%%%



%%%%%%%%%%%%%%%%%%%%%%%%%%%%%%%%%%%%%%%%%%%%%%% RESUMO - INGL�S %%%%%%%%%%%%%%%%%%%%%%%%%%%%%%%%%%%%%%%%%%%%%%



%%%%%%%%%%%%%%%%%%%%%%%%%%%%%%%%%%%%%%%%%%%%%%% INSER??O P?GINA EM BRANCO %%%%%%%%%%%%%%%%%%%%%%%%%%%%%%%%%%%%%%%%%%%%%%


%\thispagestyle{empty}
%\newpage
%%%%%%%%%%%%%%%%%%%%%%%%%%%%%%%%%%%%%%%%%%%%%%% RESUMO %%%%%%%%%%%%%%%%%%%%%%%%%%%%%%%%%%%%%%%%%%%%%

%%%%%%%%%%%%%%%%%%%%%%%%%%%%%%%%%%%%% CONFIGURA??ES DOS ?NDICES %%%%%%%%%%%%%%%%%%%%%%%%%%%%%%%%%%%%%
%\clearpage
%\thispagestyle{empty}

\listoffigures % ?ndice de Figuras
\listoftables % ?ndice de Tabelas

%%%%%%%%%%%%%%%%%%%%%%%%%%%%%%%%%%%%%%%%%%%%%%% INSER??O P?GINA EM BRANCO %%%%%%%%%%%%%%%%%%%%%%%%%%%%%%%%%%%%%%%%%%%%%%
\cleardoublepage


%%%%%%%%%%%%%%%%%%%%%%%%%%%%%%%%%%%%% LISTA DE ABREVIATURAS %%%%%%%%%%%%%%%%%%%%%%%%%%%%%%%%%%%%%


 

\begin{titlepage}
{
	\vspace{0.11\textheight}
	\raggedleft%
	\textbf{\fontsize{16pt}{5mm}\selectfont\bfseries{Siglas}}
	\vspace{0.05\textheight}
}
{
	\begin{tabbing}
		\hspace*{0.5cm}\=\hspace{2.5cm}\= \kill
		
		% Exemplo de lista de lista de abreviaturas
		\> API \> \textit{Application Programming Interface} - Interface de Programa��o de Aplicativos \\
		\> OCR \> \textit{Optical Character Recognition} - Reconhecimento �tico de Caracter \\
		\> TTS \> \textit{Text-To-Speech} - Texto-Para-Fala \\
		\> IoT \> \textit{Internet of Things} - Internet das Coisas \\
		\> IDE \> \textit{Integrated Development Environment} - Ambiente de Desenvolvimento Integrado \\
		\> ETA \> \textit{Electronic Travel	Aid} - Subs�dio Eletr�nico de Percurso \\
		\> SaaS \> \textit{Software as a Service} - Software como Servi�o \\
		
		
	\end{tabbing}
}
\end{titlepage}


			

\cleardoublepage
%%%%%%%%%%%%%%%%%%%%%%%%%%%%%%%%%%%%% CONFIGURA��ES DOS �NDICES %%%%%%%%%%%%%%%%%%%%%%%%%%%%%%%%%%%%%
%\usepackage{fancyhdr}


\pagestyle{fancy}
\fancyhf{} % clear all header and footer fields
\fancyhead[RO, LE] {\thepage}

\fancypagestyle{plain}{\pagestyle{fancy}}




\setcounter{page}{19}
\tableofcontents % �ndice Geral

